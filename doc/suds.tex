%
% To make the PostScript document:
%
%	$ touch sudsndx.dats	# empty sorted index file to start
%	$ latex suds.tex
%	$ texindex sudsndx.dat	# makes the sorted index file sudsndx.dats
%	$ latex suds.tex
%	$ dvips -o suds.ps suds.dvi
%

\documentstyle [twoside]{suds}

\setcounter{tocdepth}{1}	% Only parts and sections in toc

\tableofcontents
\doindex

%
% Heavier underscores
%
\def\_{\leavevmode \kern.06em \vbox{\hrule height 0.8pt width 0.3em}}

%
% Macros of use here.
%
\newcommand{\Suds}{{\caps Suds}}
\newcommand{\suds}{{\caps suds}}
\newcommand{\headsuds}{SUDS}
\newcommand{\minus}{${}-{}$}
\newcommand{\sub}[1]{$_{#1}$}
\newfont{\bfss}{cmssbx10 scaled \magstep3}

\title{	SUDS: \\
	The System for User-editing and \\
	Display of Soundings}
\author{Chris Burghart \\
	Research Data Program \\
	Atmospheric Technology Division \\
	National Center for Atmospheric Research}

\begin{document}
\maketitle

\part {Introduction}
The System for User-editing and Display of Soundings, or \suds, is the RDSS
software package for sounding manipulation.  This document describes the
program and its usage.

\section {Running and Stopping the Program}
\index{running \suds}
\Suds\ can be run by typing the command:
\begin{example}
	suds
\end{example}
at the system level prompt.  The program will respond with its own 
prompt, |->|, at which point the user can begin entering the commands
described in this document.

The command to leave \suds\ is:
\ndxentry{exit}{\tt exit}
\index{stopping \suds}
\begin{example}
	exit
\end{example}
Simply type this command at the \suds\ prompt, and the program will terminate.

\section {Concept}
\index{purpose of \suds}
\Suds\ is being developed to provide RDSS users with the capability to examine
and modify soundings from various systems.  The current functionality of
\suds\ includes these abilities in very simple form, but work on the
program continues, and many enhancements will be added as they are developed.
The order and extent of improvements will be very dependent on feedback from
users; the more requests there are for a given change, the more likely the 
change is to occur.  Let me know what you want to see.

\section {The User Interface}
\index{user interface}
\Suds\ is driven by the RDSS User Interaction Package, the same interface used
by the {\caps robot} program.  At the bottom level, it is a command based 
system, with the capability to be adapted to menus.  Only the command interface
to \suds\ is described here; the menu interface, when it is developed, will be
documented separately.

\section {Structure of the Manual}
\index{structure of the manual}
This manual is divided into five parts.  The first part is the introduction
which you are now reading.  Part~2 explains the procedures for accessing
soundings for use in the program.  Part~3 covers the various commands which
are used in generating plots.  Part~4 explains how to edit a sounding and 
how to save a sounding once it has been modified.  Finally, Part~5 contains a
glossary of all of the commands available in \suds.

\index{command syntax}
Command syntax is displayed in this manual as shown below, where the |skewt|
command is used as an example:
\begin{example}
	skewt [plot] [|id-name|\sub{1} |id-name|\sub{2}\ldots |id-name|\sub{n}]
\end{example}
Words which are shown in |typewriter type|, like |skewt| and |plot| above,
are keywords and should be typed
as shown.  The parts of the command shown in {\pf italic typewriter type},
like {\pf id-name}\sub{1} and {\pf id-name}\sub{2} are
parameters for the command.  Appropriate values or names should be 
substituted in these positions.   The parts of the syntax which are enclosed
in square brackets, as the keyword |plot| is above, are optional parts of
the command.  In the case of optional keywords, they are simply provided as 
filler words which can make the command more descriptive.  In the case of 
optional parameters, like {\pf id-name}\sub{1}\ldots {\pf id-name}\sub{n}, 
they refer to parameters which may be supplied if the user wishes, but 
are not required.

\section{Flags}
\mainindex{flags}
A number of things which \suds\ does may be changed with the use of logical
flags.  Throughout the document, these named flags and their functions will
be described.  The user can change the values of the flags using the ||set||
command:
\begin{example}
	set |fname| true
	  {\rm --or--}
	set |fname| false
\end{example}
where {\pf fname} is the name of the flag to be changed.
The default values for the flags are included in the descriptions of each,
and the ||show flags|| command:
\begin{example}
	show flags
\end{example}
will display the current state of all program flags.


\part {Accessing Soundings}
\index{accessing soundings}
Before anything can be done with a sounding in \suds, the sounding must
be made available to the program.  
This part explains the commands available for bringing soundings into the
system and for making copies of them if necessary.

\section {Reading a Sounding File}
\index{reading a sounding file}
\subindex{soundings}{reading from a file}
Soundings are generally stored on disk, with each sounding in its own file.
The file for a given sounding must be declared to \suds\ before it can be used
in the program.  The command ||file|| is used for this purpose; its
syntax is:
\begin{example}
	file [name] |filename| [type] |filetype| [id] |sounding-id|
\end{example}
The {\pf filename} is just the system name of the file from which the sounding
is to be taken.  \subindex{soundings}{format of}\index{format, sounding}The 
format of the sounding file is indicated by the {\pf filetype} parameter.  
Legal formats are shown in Table~\ref{tbl-formats}.  New formats will be added
as necessary.  The final parameter to the |file| command 
is the \subindex{soundings}{identifiers (id's) of}\index{identifier, sounding} 
{\pf sounding-id}, which is the 
name which will be assigned to the chosen sounding for the remainder of the 
\suds\ session.  All future references to the sounding will use this 
identifier.

\begin{table}[tp]
	\begin{center}
	\begin{tabular}{|c|l|}
		\hline
		Format		& Description \\
		\hline \hline
		\tt cape	& CaPE format \\
		\tt class	& CLASS format (both old and new) \\
                \tt drexel      & Drexel University Gale Data Center format \\
		\tt fgge	& FGGE format \\
		\tt gale	& GALE format \\
		\tt jaws	& JAWS format \\
		\tt mist	& MIST format \\
		\tt ncar	& NCAR-mobile format \\
		\tt netcdf	& netCDF (Network Common Data Format) \\
		\tt nmc		& National Meteorological Center format \\
		\tt noaa	& NOAA-mobile format \\
		\tt nssl	& National Severe Storms Laboratory format \\
		\tt nws		& National Weather Service format\\
		\tt rsanal	& RSANAL format \\
                \tt wmo		& WMO format \\ \hline
	\end{tabular}
	\end{center}
	\caption{Legal formats for the {\tt file} command.}
	\label{tbl-formats}
\end{table}

\mainindex{current default sounding}
\mainsubindex{soundings}{current default}
The sounding thus loaded will also be assigned as the current
default sounding.  Many commands in \suds\ allow the user to omit the 
specification of a sounding to be used.  In these cases, the current default
sounding will be used.

The |ncar|, |noaa|, and |rsanal| formats require that the user specify an
origin before a file of that format can be loaded.  The origin is used to
convert the site location in $(x,y)$ space into a location in
(latitude,longitude) space.  If no origin has been specified (see the
||origin|| command) when a file in one of these formats is loaded, the user 
will be prompted for an origin latitude and longitude.

Files in |wmo| format usually contain multiple soundings, but the |only|
option is provided to allow for loading soundings only from a specific site:
\begin{example}
	file [name] |filename| [type] wmo [only |site|] [id] |sounding-id|
\end{example}
Also, WMO sounding files are notorious for having bad or mis-formatted
data.  In general, \suds\ will quietly skip over most of these problems.
However, the ||wmoQuiet|| flag can be set to false to print messages for
all problems found.

\section {Creating a Copy of a Sounding}
\index{copying soundings}
\subindex{soundings}{copying}
A user may wish to make a copy of a sounding before applying some
edit command to it.  This allows experimentation without having to worry
about destroying previous work.
A copy of a sounding may be made using the ||create|| command:
\begin{example}
	create [sounding] |new-id| [from] |sounding-id|
\end{example}
where {\pf new-id} is the identifier to be assigned to the created sounding
and {\pf sounding-id} is the identifier of the sounding from which it is being
created.
\index{current default sounding}\subindex{soundings}{current default}
The created sounding will be 
established as the current default sounding.


\section{Removing Soundings}
\index{removing soundings}
\subindex{soundings}{removing}
A sounding may be removed from the list of currently available soundings
using the ||forget|| command.  The syntax of the command is:
\begin{example}
	forget |sounding-id|
\end{example}
where {\pf sounding-id} is the name of the sounding to be removed from the 
list.  The {\pf sounding-id} {\it must} be specified; no default name will
be assumed.

\section{Displaying Available Soundings}

It is often useful to see which soundings have been declared within \suds.
For this purpose, the ||show soundings|| command:
\begin{example}
	show soundings
\end{example}
has been provided.  This command will display a list of the soundings 
currently available, with their identifiers, release times, site names, and
fields.


\part{Making plots}
\label{part-plot}
One of the primary functions of \suds\ is to generate plots from sounding
data.  This part of the document contains the commands necessary for
making these displays.

\section{Output Device}
Before any plots can be generated, the device on which the plots are to be
drawn must be specified.  The ||output|| command is used for this purpose,
with syntax:
\begin{example}
	output [device] |dev-name| [type] |dev-type|
\end{example}
The parameter {\pf dev-name} is the system name of the device to be used, and
{\pf dev-type} is the type of the device.  Some device names of interest
are shown in Table~\ref{tbl-dev} and the legal types are shown in 
Table~\ref{tbl-type}.  Editing is supported only on X window devices.

\begin{table}[tp]
	\begin{center}
	\begin{tabular}{|c|l|}
		\hline
		Device		& Description \\ \hline \hline
		\tt lp		& The default printer \\
		\tt screen	& X workstation screen \\ \hline
	\end{tabular}
	\end{center}
	\caption{Common graphic output devices for the {\tt output} command}
	\label{tbl-dev}
\end{table}

\begin{table}[tp]
	\begin{center}
	\begin{tabular}{|c|l|}
		\hline
		Type		& Description \\ \hline\hline
		\tt ps		& PostScript black \& white printer \\
		\tt psc		& PostScript color printer \\
		\tt X500	& X Window $500 \times 500$ window \\
		\tt X700 	& X Window $700 \times 700$ window \\
		\hline
	\end{tabular}
	\end{center}
	\caption{Legal device types for the {\tt output} command}
	\label{tbl-type}
\end{table}

\section{Skew-t, Log p Plots}
\index{skew-t, log p plots}
\subindex{plots}{skew-t, log p}
A standard skew-t, log p (hereafter simply skew-t) plot may be generated 
in \suds\ using the ||skewt|| command.  The syntax of the command is:
\begin{example}
	skewt [plot] [|id-name|\sub{1}\ldots|id-name|\sub{n}]
\end{example}
where {\pf id-name\sub{1}\ldots id-name\sub{n}} are the identifiers of the 
soundings to be plotted.  If no soundings are specified, the current default
sounding will be used.  
\index{current default sounding}\subindex{soundings}{current default}
The last sounding plotted by this command will be established as the new 
current default sounding.  Traces of temperature and dewpoint will be plotted 
for each of the soundings, with appropriate labeling.  The number of soundings 
which may be specified for one plot is limited to three. 

\subsection{Skew-t flags}
In addition to the thermodynamic data plotted in the skew-t plots, a winds
profile and traces to follow a lifted surface parcel can be displayed.  These
two options are controlled by the flags ||winds|| and ||lift||, respectively.
By default, both of the flags are set to true.  The display of the lifted 
surface parcel will be affected by the value of the ||mli|| flag, which 
controls whether analyses are performed using the standard lifted index or the
modified lifted index.  The |mli| flag is described in the {\bf Analysis of
Soundings} section. 

\index{moist adiabats, labeling}
Another flag affecting skew-t plots is the ||theta_w|| flag.  On standard
Air Force skew-t, log-p charts, moist adiabats are labeled by their
wet-bulb potential temperature, $\theta_w$.  With the |theta_w| flag
set to true (the default), \suds\ draws moist adiabats in this fashion.
When |theta_w| is false, however, equivalent potential temperature
will be used to label moist adiabats.

Finally, the temperature data shown on the skew-t plot can be set to
either ambient temperature or virtual temperature using the ||vt|| flag.
Be default, ambient temperature is used, but virtual temperature will be
plotted if |vt| is true.  Note that the |vt| flag also affects analysis
results.  See the |analysis| command below.

\subsection{Skew-t plot limits}
The default plot limits on a \suds\ skew-t plot should be sufficient to show
all points on most soundings.  It is often desirable, however, to change these
limits in order to fill more of the plot or to emphasize a portion of a 
sounding.  The commands ||skewt plimits|| and ||skewt tlimits|| allow the 
user to scale skew-t plots as desired.

\ndxentry{skewt plimits}{\tt skewt plimits}
The pressure limits of skew-t plots are set with the command:
\begin{example}
	skewt plimits [max] |p\_hi| [min] |p\_lo|
\end{example}
where {\pf p\_hi} and {\pf p\_lo} are the upper and lower pressure limits
in mb, respectively.  The default pressure limits in effect at program
initiation are 1050-100 mb.

\ndxentry{skewt tlimits}{\tt skewt tlimits}
Similarly, temperature limits may be changed using the command:
\begin{example}
	skewt tlimits [min] |t\_lo| [max] |t\_hi|
\end{example}
where {\pf t\_lo} and {\pf t\_hi} are the temperature limits in $^\circ$C.
The limits specified are the temperature limits for the bottom level (highest
pressure) of the plot.  The temperature is skewed in the plots such that 
the temperature in the lower left corner is the same as the temperature in 
the upper right corner.  Hence, the temperatures at the top of the plot extend
from 2{\pf t\_lo}\minus{\pf t\_hi} to {\pf t\_lo} $^\circ$C.  The default
values of {\pf t\_lo} and {\pf t\_hi} are -40 and 35, respectively.

\subindex{winds}{vector/barb interval}
By default, a barb or vector is drawn for every wind point in a sounding.
With the ||windstep|| command, the interval between drawn points may be set.
The syntax is:
\begin{example}
	windstep |n|
\end{example}
where {\pf n} is the step in sounding points between drawn barbs or vectors.
Setting {\pf n} to two will cause every other point to be used, setting it
to three will cause every third point to be used, etc.

\section{Hodographs}
\index{hodographs}
\subindex{plots}{hodographs}
A standard hodograph winds plot may be generated for a sounding using the 
||hodograph|| command:
\begin{example}
	hodograph [mark |m|] [step |s|] [top |t|] [|id-name|\sub{1}\ldots|id-name|\sub{n}]
\end{example}
where {\pf id-name\sub{1}\ldots id-name\sub{n}} are the identifiers of the 
soundings to be plotted.  If no soundings are specified, the current default
sounding will be used.  Up to three soundings may be shown in the same 
hodograph.  The winds for each sounding specified will be shown as a trace 
on an $x$-$y$ plane.

The optional keywords |mark|, |step|, and |top|, affect how the hodograph 
plots are drawn.  By default, every wind point in the sounding is used, and
a label is placed on the hodograph every 1~km MSL.  The |mark| keyword
may be used to change the label spacing to every {\pf m}~km MSL.
The |step| keyword causes the hodograph to be drawn with a point every 
{\pf s}~km, rather than drawing every point in the sounding.  When
|step| is specified, every point drawn will be labeled.  The |top| keyword can
be used to stop the hodograph at {\pf t}~km MSL.  By default, plots
are drawn up to 20~km MSL.

If AGL altitudes are desired instead of MSL on hodographs, the ||hodo_msl||
flag can be set false.  Be default it is true, meaning that MSL altitudes
will be used.

\section{Cross-section Plots}
\index{cross-sections}
\subindex{plots}{cross-sections}
Vertical cross-section plots (spatial or time-height) are generated in \suds\
using the ||xsect|| family of commands.  Two of the commands affect both
spatial and time-height plots.  The first of these is ||xsect use||, which
is used to specify the soundings to use for subsequent cross-section plots.
The syntax for the command is:
\begin{example}
	xsect use |snd\sub{1} snd\sub{2} \ldots|
\end{example}
where the parameters {\pf snd\sub{n}} are the identifiers of soundings to use.
There is no maximum number of soundings to use, but at least two soundings
must be specified with the |xsect use| command.  The |xsect use| command must
appear before cross-section plots can be generated.

By default, altitude is used as the vertical scale of cross-section plots.
This can be changed with the ||xsect vscale|| command:
\begin{example}
	xsect vscale |fld|
\end{example}
where {\pf fld} is the name of the field to use for the vertical scale.
Only |pres| or |alt| may be used as the vertical scale field.

\subsection{Spatial cross-sections}
\index{spatial cross-sections}
\subindex{plots}{spatial cross-sections}
For spatial cross-section plots, the surface endpoints of the vertical 
plane must be given before the cross-section can be generated.  These 
endpoints are specified using the ||xsect from - to -|| command 
which has syntax:
\begin{example}
	xsect from |left-endpoint| to |right-endpoint|
\end{example}
The endpoints can be specified in either of two ways.  The (x,y) location
of the endpoint can be used:
\begin{example}
	|xpos ypos|
\end{example}
where {\pf xpos} and {\pf ypos} are given in km and are interpreted relative
to the origin current at the time the plot is generated (see the |origin| and
|show origin| commands).  Alternatively, the launch point of a sounding
can be specified as an endpoint:
\begin{example}
	site |sounding-id|
\end{example}
The |site| keyword is required and {\pf sounding-id} is the identifier of
the sounding to use.

Finally, the actual plot of a spatial cross-section is generated using the
command:
\begin{example}
	xsect |fldname|
\end{example}
where {\pf fldname} is the name of the field to use for the plot.

Two special fields are available for plotting in spatial cross-sections,
||u_prime|| and ||v_prime||.  The $u'$ field is the wind component parallel
to the plane of the cross-section, with the positive vector pointing to
the right.  The $v'$ field is the wind component perpendicular to the
plane of the cross-section, with the positive vector pointing into the plot.

\subsection{Time-height cross-sections}
\index{time-height cross-sections}
\subindex{plots}{time-height cross-sections}
Time-height cross-section plots are generated with the ||xsect time-height||
command, which has syntax:
\begin{example}
	xsect time-height |fldname|
\end{example}
The parameter {\pf fldname} is the name of the field to plot.  The time
limits of the plot will be generated automatically based on the earliest
and latest times from all of the soundings being used for the plot.

\section{Foote Charts}
\index{Foote charts}
\subindex{plots}{Foote charts}
The Foote chart was first described by Foote (1984) and provides a means for 
predicting whether an obstacle (e.g., a gust front) of a given height will
initiate convection in an air mass.  Wilson and Mueller (1987) added the parcel
lifted index (PLI) to the chart as an indicator for intensity of initiated
convection.  A Foote chart with PLI values may be generated in \suds\ with the
||foote|| command:
\begin{example}
	foote [|id-name|\sub{1}\ldots|id-name|\sub{n}]
\end{example}
where {\pf id-name\sub{1}\ldots id-name\sub{n}} are the identifiers of the 
soundings to be plotted.  If no soundings are specified, the current default
sounding will be used.  Up to three soundings may be displayed in the same
Foote chart.

\section{X-Y Plots}
Plots of one sounding field against another can be made using the ||xyplot|| 
command, which has syntax:
\begin{example}
	xyplot |xfld| |yfld| [|id-name|\sub{1}\ldots|id-name|\sub{n}]
\end{example}
The fields to be displayed along the x- and y-axes are {\pf xfld} and 
{\pf yfld}, respectively.  The {\pf id-name} parameters are the sounding(s)
to be used in making the plot.  Up to three soundings can be displayed in
on x-y plot.  If no sounding identifier is specified, the current default
sounding will be used.  Reasonable default plot limits exist for each
field, but they can be changed using the |limits| command.


\section{Color Handling}
\index{color handling}
Any of the colors used by \suds\ in its plots can be changed by the user
with the ||color|| command:
\begin{example}
	color |index-name| |color-name|
\end{example}
where {\pf index-name} is the name describing the ``function'' of the color
being changed.  Table~\ref{tbl-clrndx} shows the index names and their default 
color values.  Each name describes its own use; data traces are drawn using
colors |data1|--|data9|, backgrounds are drawn using |bg1|--|bg4|.  Even
the program's notions of |black| and |white| can be changed.  ``White''
annotations on a plot can be made to actually come out red and ``black'' areas
can be made lime green.  For most output devices, the |black| index also 
controls the screen background color.

\begin{table}[tp]
	\begin{center}
	\begin{tabular}{|c|l|c|}
		\hline
		Index name	& Usage			& Default color \\ 
		\hline\hline
		\tt data1	& First data color 	& red \\
		\tt data2	& Second data color	& cyan \\
		\tt data3	& Third data color	& magenta \\
		\tt data4	& Fourth data color	& medium slate blue \\
		\tt data5	& Fifth data color	& yellow \\
		\tt data6	& Sixth data color	& green \\
		\tt data7	& Seventh data color	& gold \\
		\tt data8	& Eighth data color	& dark orchid \\
		\tt data9	& Ninth data color	& pale green \\
		\tt bg1		& Background drawing	& gray60 \\
		\tt bg2		& Background drawing	& gray45 \\
		\tt bg3		& Background drawing	& steel blue \\
		\tt bg4		& Background drawing	& sienna \\
		\tt black	& Screen color		& black \\
		\tt white	& Bright annotation color & white \\
		\hline
	\end{tabular}
	\end{center}
	\caption{Legal color index names for the {\tt color} command.}
	\label{tbl-clrndx}
\end{table}

The {\pf color-name} parameter is a color name from the list in 
Appendix~\ref{color-names}.  All of the primary and secondary colors are in the
list, as well as a number of other colors ranging from aquamarine to wheat.
Case is important for these color names, and names with spaces in them must
be placed in quotes.

If the named colors are insufficient, the user may specify a custom color
using:
\begin{example}
	color |index-name| rgb |redval| |greenval| |blueval|
\end{example}
where {\pf redval}, {\pf greenval}, and {\pf blueval} are the relative red,
green, and blue intensities of the color in the range 0.0--1.0.  The 
{\pf index-name} parameter can take the same values as in the standard
|color| command above.

A listing of the current color settings can be generated with the
||show colors|| command:
\begin{example}
	show colors
\end{example}
which takes no parameters.  A tabular listing of the color indices with their
current values will be printed on the user's terminal, with the color name
and the RGB values displayed.  If a graphic output device has been specified,
a plot will also be generated on that display, with some text written in each
of the colors.

\section{Plot origin}
\subindex{plots}{origin for}
The origin used to convert between (latitude,longitude) and 
($x,y$) coordinates is set using the ||origin|| command:
\begin{example}
	origin |lat| |lon|
\end{example}
where {\pf lat} and {\pf lon} are the latitude and longitude of the chosen
origin.  The values should be given in decimal degrees (e.g., 
$39^\circ 30'00''$ is specified as 39.500).  Standard sign conventions are 
used, so north latitudes and east longitudes are positive.  This means that
{\em west longitudes must be entered as negative values}.

The current value of the origin may be examined with the ||show origin||
command:
\begin{example}
	show origin
\end{example}
which takes no parameters.

\section{Winds scaling}
\subindex{winds}{scaling}
The scaling of winds values on ||skewt|| and ||hodograph|| plots may be 
changed with the ||wscale|| command:
\begin{example}
	wscale |val|
\end{example}
where {\pf val} is an estimate of the maximum wind speed (in ms$^{-1}$) to
be plotted.  Winds less than {\pf val} are assured to fit within the bounds
of the plot.  The default wind scaling value is 25.0 ms$^{-1}$.

\section{Field Limits}
\index{limits}
Four plot limits are associated with each field understood by \suds:
bottom, top, center, and step.  The bottom and top are the plotting bounds
used for x-y plots and for the vertical scales of skew-t and cross-section
plots.  Bottom sets the limit for the bottom (or left) side of the plot,
and top sets the limit for the top (or right) side of the plot.  The center
and step values are used for contours in cross-section plots.  Center is
the first contour value and step is the spacing to use between contours on
either side of the center.  The center value also determines how color will
be used if the display device supports it.  Color index |data5| is used for
the center contour, and the colors are used down to |data1| and up to
|data9| for contours below and above the center value respectively.
Contours outside this range of nine colors will be plotted using color
index |bg1| (medium gray by default).

\subindex{limits}{examining}
The current limits for all fields may be shown using the ||show limits|| 
command:
\begin{example}
	show limits
\end{example}
which takes no parameters.  Each known field is listed along with its 
bottom, top, center, and step values.

\subsection{Changing bottom and top limits}
\subindex{limits}{bottom and top}
The bottom and top limits for a field may be changed 
with the ||limits|| command:
\begin{example}
	limits |fld| |bottom| |top|
\end{example}
where {\pf fld} is the name of the field to be affected, {\pf bottom} is
the bottom value to use, and {\pf top} is the top value to use.

\subsection{Changing contour limits}
\index{contour limits}
\subindex{limits}{contour}
The center value and spacing used in contour plots may be changed on a
field-by-field basis with the ||conlimits|| command:
\begin{example}
	conlimits |fld| |center| |step|
\end{example}
where {\pf fld} is the name of the field to be affected, {\pf center} is
the center contour value, and {\pf step} is the spacing to use between 
contours.  The possible contour values used for the chosen field are determined
by the formula ${\pf center}\pm n\times{\pf step}$ where $n$ can be any 
integer.  


\part{Editing Soundings}
\index{editing}
\subindex{soundings}{editing}
The process of editing in \suds\ consists of three basic functions.  
The first step is to specify the particular sounding and field
to be edited, using the ||select|| command.  Once a field has been selected,
it remains so defined until another |select|
command is given.  If the chosen field is in the current plot, a cursor
will appear on the plot, over the first point of the chosen field.
The second step of editing involves movement through the data to mark a point 
or region to be edited. 
Third, an editing function is chosen which is then
applied to the point or points which have been selected.  These procedures 
may then be repeated as necessary to perform the desired modifications.
Generally, the second and third steps are repeated until the user is finished
editing a field, then the |select| command is used to change fields and
the process is repeated.  Edited soundings may optionally be written to
disk files to save them for later editing or analysis.

The \suds\ commands used in editing are described in detail in the following
sections.

\section{Selecting the Field to be Edited}
\subindex{editing}{selecting a field for}
Once one or more soundings have been made known to \suds, a field may be 
chosen for editing by using the ||select|| command.  The syntax of this 
command is:
\begin{example}
	select [field] |fld-name| [[sounding] |sounding-id|]
\end{example}
where {\pf fld-name} is the name of the field to edit and 
{\pf sounding-id} is the name of the sounding being edited.  If no sounding 
is specified, the current default sounding will be assumed.  Legal field 
names are shown in Appendix~\ref{app-fields}.

If a field which is in the current plot is selected, a cursor will appear 
over the first point of that field.

\section{Pointer Movement}
\index{pointer movement}
\subindex{editing}{pointer movement}
The current position within the data of the currently selected field is 
referred to as the pointer position.  The |select| command will position the 
pointer on the first point of the chosen field; it may then be moved using 
the commands ||up|| and ||down||, which have syntax:
\begin{example}
	up [|count|]
	  {\rm --and--}
	down [|count|]
\end{example}
The optional parameter {\pf count} tells how many points to move up or
down the trace; it defaults to one if not specified.  Note that pointer
movement in \suds\ is handled only through program commands.  Use of the
Ramtek trackball, as is done in the RDSS radar editor, will move the cursor on
the screen, but will not be recognized by \suds.

\section{Examining Data Values}
\index{values, examining}
\index{data values, examining}
\subindex{editing}{examining data values}
When the pointer is on a given point, it is often useful to know the exact
value of the point.  The command:
\ndxentry{examine}{\tt examine}
\begin{example}
	examine
\end{example}
will show the current pointer position, as well as the positions of points  
in the region of the pointer, up to three points to either side.
The |examine| command takes no parameters.

\section{Region Selection}
\index{region selection}
\subindex{editing}{region selection}
Some editing functions can affect a contiguous set of points, or a region.
The current editing region is defined by its two endpoints, one of which is 
established by issuing the command:
\ndxentry{mark}{\tt mark}
\begin{example}
	mark
\end{example}
when the pointer is over the desired point.  The pointer may then be moved,
and the second endpoint of the editing region is defined as the current 
pointer location.  For fields in the current plot, \suds\ will display the 
current region by overwriting the edited trace in white.  A |mark| will be 
deleted when another |mark| command is issued or when an editing function uses 
the current region.

\section{Editing Functions}
The editing functions described here directly modify the selected sounding and
field.  Some functions affect only the one point marked by the pointer, while
others will affect all of the points in the current editing region.  The
descriptions below will explain the results of each command.  After each of 
the commands below is used, the current plot will be redrawn if necessary to 
reflect the changes effected by the editing.

\subsection{Deleting individual points}
\subindex{editing}{deleting individual points}
A single point may be removed from a sounding using the command:
\ndxentry{erase}{\tt erase}
\begin{example}
	erase
\end{example}
which takes no parameters.  The command deletes the point which is currently 
under the pointer.  The pointer will be moved to the next point up (next point
down if the point deleted is at the top).  The |mark| position, if any, will
remain unchanged unless it was on the deleted point, in which case the 
position will be nullified.

\subsection{Deleting regions of points}
\subindex{editing}{deleting regions of points}
All of the points in the current editing region, exclusive of the endpoints,
may be removed from the sounding with the command:
\ndxentry{cut}{\tt cut}
\begin{example}
	cut
\end{example}
which takes no parameters.  A |cut| command leaves the pointer in its current
position and nullifies the ||mark|| position.  If |cut| is used and there
is no current editing region (i.e., no |mark| has been placed or the pointer
is on top of the mark), an error will result and the sounding will be
unchanged.

\subsection{Replacing individual data values}
\subindex{editing}{replacing individual data values}
\index{replacing individual data values}
Occasionally, it is useful to change a given point in a sounding.  The 
value associated with the point under the current pointer location may be
changed with the ||newvalue|| command:
\begin{example}
	newvalue |val|
\end{example}
where {\pf val} is the new value to be substituted for the current one.  
Other parts of the sounding will be unaffected and the pointer will remain 
on the same point.

\subsection{Inserting new data points}
\subindex{editing}{inserting data points}
\index{inserting data points}
\Suds\ also allows the user to insert new data points into a sounding 
which is being edited.  With the ||insert|| command, a list of fields and
associated values can be specified to create a new point, which is then 
inserted either above or below the current pointer location.
\begin{example}
	insert |where| |fld-name\sub{1}| |val\sub{1}| [|fld-name\sub{2}| |val\sub{2}| \ldots]
\end{example}
The first parameter, {\pf where}, is one of the keywords |above| or |below|, 
depending on where the point is to be inserted.  The following parameters 
should be ({\pf fld-name}, {\pf val}) pairs to use in creating the new 
point.  As an example, if a user wished to insert a point above the current 
position with pressure 500 mb, temperature -5$^\circ$C, and dewpoint 
-10$^\circ$C, the command would be:
\begin{example}
	insert above pres 500 temp -5 dp -10
\end{example}
The specified point would be inserted immediately above the current point 
in the list of data points.  

NOTE: Only the position of the pointer is used for this edit procedure; the
currently selected field is only affected if it is explicitly included in 
the list of fields in the |insert| command.

\subsection{Data Thresholding}
\subindex{editing}{thresholding}
\index{data thresholding}
\index{thresholding}
A commonly needed function is the ability to remove points from a field based
on the values from another field (or perhaps the same field).  This capability
is provided in \suds\ by the ||threshold|| command, which has syntax:
\begin{example}
	threshold [field] |f| [on] |t| |c| [[and] |t|\sub{2} |c|\sub{2} \ldots] [sounding |id-name|]
\end{example}
The command removes points from target field {\pf f} for which the
corresponding points in field {\pf t} fit the given criterion {\pf c}.
If no sounding is specified by {\pf id-name}, the current default sounding
will be used.  Multiple threshold fields and criteria may be specified, if
desired.  In the case of multiple criteria, a target point will be removed
only if {\it all} criteria are met.

Criteria are specified by comparative expressions or by the keyword |bad|.
Legal criteria are shown in Table~\ref{tbl-crit}.  As an example, consider
a case in which sounding |test| has a some glitched wind speed data.  The
user decides to remove all points with wind speed greater than 75 $ms^{-1}$
in order to get rid of the glitches.  The command to accomplish this would
be:
\begin{example}
	threshold wspd on wspd > 75.0 sounding test
\end{example}
Notice that there is a space on either side of the ``|>|''; this is
necessary in order for the command to be parsed properly.  As a second
example, consider a CLASS system sounding, where a pressure quality field,
|qpres|, is provided.  In this case the user chooses to remove pressures
that have a quality number greater than 1.5, but does not want to remove
points for which the quality number is 77 (which is a special value for
CLASS soundings).  The |threshold| command to do this for the current
default sounding is:
\begin{example}
	threshold pres on qpres > 1.5 and qpres <> 77
\end{example}
Multiple criteria can be combined like this in many different ways, and can
involve separate fields for each criterion.  Up to ten different criteria can
be included in a single |threshold| command.

\begin{table}[tp]
	\begin{center}
	\begin{tabular}{|c|c|}
		\hline
		\tt = \pf val	& \tt <> \pf val \\
		\tt > \pf val	& \tt < \pf val \\
		\tt >= \pf val	& \tt <= \pf val\\
		\tt bad 	& \\ \hline
	\end{tabular}
	\end{center}
	\caption{Legal criteria for the {\tt threshold} command}
	\label{tbl-crit}
\end{table}

Another user has edited the temperature field in the current default sounding,
removing some bad points.  Since dewpoint is derived in part from temperature,
the corresponding data in the dewpoint field should also be removed.  Rather
than editing the dewpoint in a similar fashion as the temperature, the user
can simply remove data from the dewpoint field corresponding to the now deleted
points in the temperature field using the command:
\begin{example}
	threshold dp on temp bad
\end{example}
After this command is executed, all dewpoint data which do not have 
corresponding good points in the temperature field will have been deleted.

Note that this command differs from other editing commands in that the fields
and sounding to be used are specified as part of the command.  The last 
|select| command and the current editing region do not affect and are not
affected by the |threshold| command.

\subsection{Extending Values to the Surface}
\subindex{editing}{extending data}
\index{extending data to the surface}
Occasionally, it is useful to make data follow a dry adiabat or a constant
mixing ratio from the surface to a chosen level.  This can be accomplished
using the ||extend|| command:
\begin{example}
	extend
\end{example}
which takes no parameters.  If the selected field is temperature, the
|extend| command will cause the temperature points from the current
position down to the surface to be modified to follow the dry adiabat which
intersects the current point.  The effect is to ``extend'' a dry adiabat
from the pointer to the surface.  If dewpoint is the currently selected
field, then a similar procedure is used, except that the data will be
forced to follow a line of constant mixing ratio.  If any field other that
temperature or dewpoint is selected, an error message will be issued.

\section{Saving Edited Soundings}
\index{saving an edited sounding}
\subindex{editing}{saving an edited sounding}
After a sounding has been edited, it is often desirable to save it for future
reference or continued editing.  The ||write|| command allows the user
to put a sounding into a file which can be read again later.  The syntax of 
the command is:
\begin{example}
	write [file] |filename| [[from] |sounding-id|]
\end{example}
where {\pf sounding-id} is the identifier of the sounding to be
written and {\pf filename} is the system name of the file to be created.
If no sounding is specified, the current default sounding will be written.
The type of the created file is ||CLASS||; this is the type which needs
to be specified if the file is read in again using the |file| command.

Soundings can also be saved as netCDF files using the ||netcdf|| command,
which is structured like the |write| command:
\begin{example}
	netcdf [file] |filename| [[from] |sounding-id|]
\end{example}
Files written in this way can be used in other programs which understand
netCDF.

\section{Analysis of Soundings}
\index{analyzing a sounding}
\subindex{soundings}{analysis of}
Of course, determination of characteristics of a sounding is important.
Currently, \suds\ provides only the ||analyze|| command to perform calculations
on a sounding.  The syntax of the command is:
\begin{example}
	analyze [to |filename|] [|sounding-id|]
\end{example}
where {\pf sounding-id} is the identifier for the sounding of interest.  If no
sounding is specified, the current default sounding will be used.  
Normally, the analysis is written only to the user's terminal screen.  If
the optional |to| qualifier is used, however, the analysis will also be put 
into a disk file with name {\pf filename}.


\index{lifting condensation level}
\index{LCL}
\index{lifted index}
\index{modified lifted index}
\index{level of free convection}
\index{LFC}
\index{CAPE}
\index{bulk Richardson number}
\index{mean layer vector wind}
\index{MLVW}
The |analyze| command will display:
\begin{itemize}
	\parskip=0pt
	\item Surface potential temperature and virtual potential temperature
	\item Surface mixing ratio
	\item Lifted index level temperature, potential temperature, virtual 
		temperature, and virtual potential temperature (the lifted
		index level is either 500 or 400 mb.  See the description of 
		the ||mli|| flag below)
	\item Pressure of the lifting condensation level (LCL)
	\item Lifted index or modified lifted index
	\item Pressure of the level of free convection (LFC)
	\item Positive area, in J kg$^{-1}$, below the LFC
	\item Negative area, in J kg$^{-1}$, below the LFC
	\item CAPE (Positive area, in J kg$^{-1}$, above the LFC)
	\item Negative area, in J kg$^{-1}$, above the LFC
	\item Shear over the lowest 6 km of the sounding
	\item Bulk Richardson number
	\item MLVW (Mean Layer Vector Wind)
\end{itemize}
and an equivalent ``forecast'' analysis based on values at a user-selected
pressure, rather than the surface values.

By default, surface values used for the analysis are actually averages over
the lowest 50 mb of the sounding.  To change the depth of the mixed layer,
the command ||mixdepth|| can be used:
\begin{example}
	mixdepth |depth|
\end{example}
where {\pf depth} is the desired depth of the mixed layer in mb.  A value
of zero will defeat averaging and the absolute surface values will be used.

The pressure used for the forecast analysis is 700 mb by default, but may
be changed with the ||forecast|| command:
\begin{example}
	forecast |forecast-level|
\end{example}
The {\pf forecast-level} is the pressure in mb to be used as the basis
for the forecast analysis.

Normally, analyses will be performed using ambient temperature to calculate
LCL, lifted index, areas, and Bulk Richardson number.  If the ||vt|| flag
is true, however, virtual temperature will be used for these calculations.

\subsection{Lifted Index Calculation}
\index{lifted index}
\index{modified lifted index}
Lifted index calculations will be affected by the ||mli|| or ``modified
lifted index'' flag.  This flag, which is true by default, causes the
analysis to use the modified lifted index rather than the standard lifted
index.  In calculating the MLI, the parcel is lifted to 400 mb rather than
500 mb.

\subsection{Mean Layer Vector Wind}
\index{mean layer vector wind}
\index{MLVW}
The MLVW (mean layer vector wind) is the mean wind between 1000 mb and 700 mb.
The mean u and v components of winds within the pressure limits are calculated
independently, and then combined to yield the MLVW.  The user can change the
pressure limits used to calculate MLVW with the ||mlvw|| command:
\begin{example}
	mlvw [bottom] |bot-pres| [top] |top-pres|
\end{example}
where {\pf bot-pres} and {\pf top-pres} are the pressures in mb of the bottom
and top of the layer to use.  (Presumably {\pf bot-pres} will be greater 
than {\pf top-pres}).

\subsection{Adding Items to the Analysis}
An analysis can have certain custom values added to it by the user with
the ||analyze show|| command.  This command allows the specification of
a list of fields and the pressures at which they are to be evaluated. The
syntax of the command is:
\begin{example}
	analyze show |fld| [at] |pres| [|fld| [at] |pres|]\ldots
\end{example}
where each {\pf fld} is the name of a field to be displayed and the 
corresponding {\pf pres} is the pressure at which to evaluate the field.
The chosen values will be shown at the end of each analysis under the heading
``USER-REQUESTED VARIABLES.''


\part{Glossary of Commands}
The following is an alphabetical listing of all of the commands available
in \suds.

\begin{glossarylist}
\keyword{analyze [to {\pf filename}] [{\pf sounding-id}]}
	The ||analyze|| command performs an analysis on the sounding with 
	identifier {\pf sounding-id}.  The current default sounding will be
	used if none is specified.  The LCL, LFC, and some important
	areas are displayed.  If the optional qualifier |to| is used, the
	analysis will be written in to file {\pf filename}.

\keyword{analyze show {\pf fld} [at] {\pf pres} [{\pf fld} [at] {\pf pres}]\ldots}
	Field-pressure pairs can be displayed as part of an analysis using
	the ||analyze show|| command.  Each {\pf fld} parameter is a field
	name and the associated {\pf pres} is the pressure at which to
	evaluate the field.  The values are printed at the end of each
	analysis generated by the |analyze| command.

\keyword{color {\pf index-name} {\pf color-name}}
	The ||color|| command changes the color associated with the given
	{\pf index-name} to the named color specified by {\pf color-name}.

\keyword{color {\pf index-name} rgb {\pf redval greenval blueval}}
	Like the standard |color| command above, except a custom color is
	specified by its red, green, and blue values.  The parameters
	{\pf redval}, {\pf greenval}, and {\pf blueval} should be numbers in
	the range 0.0--1.0 specifying the relative red, green, and blue 
	intensities.

\keyword{conlimits {\pf fld} {\pf center} {\pf step}}
	The ||conlimits|| command allows the user to set the contour limits
	for field {\pf fld}.  The values {\pf center} and {\pf step} are
	the basis contour value and the spacing between contours for the
	chosen field.

\keyword{copyright}
	Displays the standard UCAR copyright message.

\keyword{create [sounding] {\pf new-id} [from] {\pf sounding-id}}
	The ||create|| command generates a new sounding with 
	identifier {\pf new-id} from the sounding with identifier 
	{\pf sounding-id}.  The created sounding will be established as
	the current default sounding.

\keyword{cut}
	After an editing region has been chosen, the ||cut|| command removes
	all of the points in the editing region, exclusive of the endpoints.

\keyword{down [{\pf count}]}
	The ||down|| command moves the pointer down {\pf count} points in the
	current trace.  If {\pf count} is not specified, it defaults to one.

\keyword{erase}
	The point currently under the pointer is removed by the ||erase||
	command.

\keyword{examine}
	The data values at and around the pointer are shown by the 
	||examine|| command.  Up to three points on either side of the
	pointer are shown in addition to the current location.

\keyword{exit}
	The ||exit|| command terminates \suds, returning the user to the
	system level.

\keyword{extend}
	Data below the pointer position will be forced to fit along a given
	path depending on the field selected.  Temperature, if selected,
	will be made to follow a dry adiabat, dewpoint will be set to
	a constant mixing ratio, and any other selected field will not be
	affected.

\keyword{file [name] {\pf filename} [type] {\pf filetype} [id] {\pf sounding-id}}
	Soundings may be loaded from disk files using the ||file|| command, 
	where {\pf filename} is the system name of the file, and {\pf filetype}
	is the type of file being read (e.g., |CLASS| or |NOAA|).  The
	{\pf sounding-id} is the identifier to be used in referring to the 
	sounding within the program.  The sounding accessed will become the
	current default sounding.

\keyword{forecast {\pf forecast-level}}
	The pressure used for the forecast analysis (700 mb by default), may
	be changed with the ||forecast|| command.  The {\pf forecast-level} 
	is the pressure in mb to be used as the basis for the analysis.

\keyword{foote [{\pf id-name}\sub{1}\ldots{\pf id-name}\sub{n}]}
	A Foote chart may be generated with the ||foote|| command.
	Charts for the soundings specified by {\pf id-name}\sub{1}\ldots 
	{\pf id-name}\sub{n} will be plotted.  If no sounding is specified,
	the current default sounding will be used.

\keyword{forget {\pf sounding-id}}
	A sounding may be removed from the list of available soundings using
	the ||forget|| command.  Note that the sounding to be removed must
	be given; no default will be used.

\keyword{hodograph [mark {\pf mval}] [step {\pf sval}] [top {\pf topval}] [{\pf id-name}\sub{1}\ldots{\pf id-name}\sub{n}]}
	A standard hodograph is generated by the ||hodograph|| command.
	Up to three soundings, specified by {\pf id-name}\sub{1}\ldots
	{\pf id-name}\sub{n} will be plotted on the hodograph.  If no sounding
	is specified, the current default sounding will be used.  The |mark|
	keyword changes the labeling distance from 1~km to {\pf mval}~km.
	The |step| keyword allows the user to specify that points be drawn 
	every {\pf sval}~km; by default, every wind point in the sounding is 
	drawn.  When |top| is used, the hodograph will be drawn up to
	{\pf topval}~km MSL instead of the default 20~km MSL.

\keyword{insert {\pf where} {\pf fld-name\sub{1}} {\pf val\sub{1}} %
   [{\pf fld-name\sub{2}} {\pf val\sub{2}} \ldots]}
	A data point may be inserted into the currently selected sounding
	with the ||insert|| command.  The parameter {\pf where} specifies
	whether the point is to go |above| or |below| the current pointer
	position and is followed by a list of ({\pf fld-name}, {\pf val})
	pairs to be used in making the point.

\keyword{limits {\pf fld} {\pf bottom} {\pf top}}
	The ||limits|| command allows the user to set the bottom and top
	limits for field {\pf fld}.  The values {\pf bottom} and {\pf top}
	are the values to use at the bottom (or left) and top (or right)
	sides of the plot when the selected field is plotted.

\keyword{mark}
	The beginning of an editing region is specified with the ||mark||
	command.  The command is issued when the pointer is on the desired
	starting location.

\keyword{mixdepth {\pf depth}}
	The ||mixdepth|| command is used to change the mixed layer depth used 
	in calculating surface values for sounding analyses.  The chosen
	{\pf depth} is specified in mb.  50 mb is the default mixed layer
	depth.

\keyword{mlvw [bottom] {\pf bot-pres} [top] {\pf top-pres}}
	The ||mlvw|| command changes the layer over which Mean Layer Vector
	Wind is calculated.  The pressure in mb of the bottom of the layer is
	{\pf bot-pres} and at the top of the layer is {\pf top-pres}.  By
	default, the limits are 1000 mb and 700 mb, respectively.

\keyword{netcdf [file] {\pf filename} [[from] {\pf sounding-id}]}
	The ||netcdf|| command saves the sounding specified by 
	{\pf sounding-id} into a disk file with the name {\pf filename}.
	If no sounding is specified, the current default sounding will be 
	written.  The file created has type |netcdf| (see the |file| 
	command for the usage of file type).

\keyword{newvalue {\pf val}}
	The ||newvalue|| command substitutes {\pf val} for the value currently
	associated with the point under the pointer.  The rest of the sounding
	remains unchanged.

\keyword{origin {\pf lat} {\pf lon}}
	The ||origin|| command sets the origin for conversion between
	(latitude,longitude) and ($x,y$) coordinates.  The parameters
	{\pf lat} and {\pf lon} are specified in decimal degrees.

\keyword{output [device] {\pf dev-name} [type] {\pf dev-type}}
	The graphics output device is specified with the ||output|| command.
	The parameters {\pf dev-name} and {\pf dev-type} refer to the device's
	system name and type, respectively.  See Tables \ref{tbl-dev} and 
	\ref{tbl-type} for some common devices and types.

\keyword{select [field] {\pf fld-name} [[sounding] {\pf sounding-id}]}
	A field to be edited is specified with the ||select|| command.  The
	{\pf sounding-id} is the identifier for the sounding to be used
	and {\pf fld-name} is the name of the field to be edited.  The
	selected field must have been displayed in the previous plot.
	If no sounding is specified, the current default sounding will be
	assumed.

\keyword{set {\pf flagname} {\pf value}}
	The ||set|| command is used to change the value of program flag
	{\pf flagname} to {\pf value}.  The {\pf value} can be either
	|true| or |false|.

\keyword{show colors}
	The ||show colors|| command generates listings of the current color
	settings to both the user's terminal and to the current graphics
	device.

\keyword{show flags}
	The current state of all program flags may be shown using the
	||show flags|| command.

\keyword{show limits}
	The current plot limits for all fields are displayed by the 
	||show limits|| command.

\keyword{show origin}
	The ||show origin|| command shows the current value of the origin
	used for converting between (latitude,longitude) and
	($x,y$) coordinates.

\keyword{show soundings}
	The ||show soundings|| command displays a list of the currently
	available soundings, with the identifier, time, and site for each.

\keyword{skewt [plot] [{\pf id-name}\sub{1}\ldots{\pf id-name}\sub{n}]}
	A skew-t, log p plot may be generated with the ||skewt|| command.
	The soundings specified by {\pf id-name}\sub{1}\ldots
	{\pf id-name}\sub{n} will be plotted.  If no sounding is specified,
	the current default sounding will be used.  The last sounding plotted
	by this command will become the new current default sounding.

\keyword{skewt plimits [max] {\pf p\_hi} [min] {\pf p\_lo}}
	The ||skewt plimits|| command is used to change the pressure limits
	used in |skewt| plots.  The maximum and minimum pressures to display
	are {\pf p\_hi} and {\pf p\_lo}, respectively.

\keyword{skewt tlimits [min] {\pf t\_lo} [max] {\pf t\_hi}}
	The temperature limits of |skewt| plots are set with the 
	||skewt tlimits|| command.  The minimum and maximum temperatures,
	{\pf t\_lo} and {\pf t\_hi}, apply to the temperature limits at
	the highest pressure (bottom) of the plot.  The temperature
	limits at the top of the plot are different due to the skew in the
	temperature scale.

\keyword{threshold [field] {\pf target} [on] {\pf thresh} {\pf criterion} [[sounding] {\pf id-name}]}
	The ||threshold|| command removes points from field {\pf target} 
	for which corresponding points in field {\pf thresh} meet the given 
	{\pf criterion}.  The sounding to be used is given by {\pf id-name}
	or the current default sounding will be used if none is specified.

\keyword{up [{\pf count}]}
	The pointer may be moved up the trace with the ||up|| command.  The
	number of points moved is determined by {\pf count}.  If {\pf count}
	is not specified, it defaults to one.

\keyword{xsect {\pf fldname}}
	The ||xsect|| command generates a spatial cross section plot of
	the field {\pf fldname}.  Before a cross-section can be made,
	the plot must be defined using the  |xsect use|, 
	|xsect from - to -|, and |origin| commands.

\keyword{xsect from {\pf xleft yleft} to {\pf xright to yright}}
	The surface endpoints of the plane for a spatial cross-section
	plot are given with the ||xsect from - to -|| command.
	The parameters {\pf xleft}, {\pf yleft}, {xright}, and {\pf yright}
	specify the $(x,y)$ locations of the left and right endpoints,
	respectively.  The values should be given in km, and are interpreted
	with respect to the origin current at the time of the plot.

\keyword{xsect time-height {\pf fldname}}
	The ||xsect time-height|| command is used to make a time-height
	cross-section plot.  Before the plot can be made, the soundings
	to use must be given with the |xsect use| command.

\keyword{xsect use {\pf snd\sub{1} snd\sub{2} \ldots}}
	The soundings to use in cross-section plots (spatial or time-height)
	are specified with the ||xsect use|| command.  The parameters
	{\pf snd\sub{n}} are the identifiers for the chosen soundings.  There
	is no maximum number of soundings to use, but at least two must
	be given.

\keyword{xsect vscale {\pf fldname}}
	The vertical scale for cross-section plots may be changed with
	the ||xsect vscale|| command.  The {\pf fldname} parameter 
	specifies the field to be used for the vertical scale, either 
	|pres| or |alt|.

\keyword{xyplot {\pf xfld} {\pf yfld} [{\pf id-name}\sub{1}\ldots{\pf id-name}\sub{n}]}
	One field can be plotted against another using the ||xyplot|| command.
	The parameters {\pf xfld} and {\pf yfld} tell the fields to plot
	against the x- and y- axes, respectively.  The soundings specified
	by {\pf id-name}\sub{1}--{\pf id-name}\sub{n} will be used for the
	plot, or the current default sounding will be used if no soundings
	are given.

\keyword{write [file] {\pf filename} [[from] {\pf sounding-id}]}
	The ||write|| command saves the sounding specified by 
	{\pf sounding-id} into a disk file with the name {\pf filename}.
	If no sounding is specified, the current default sounding will be 
	written.  The file created has type |CLASS| (see the |file| 
	command for the usage of file type).

\end{glossarylist}



\appendix

\part{Field Derivation}

Formulas for most of the derivations used by \suds\ are extracted from {\sl
Formulation of Output Parameters for PAM II CMF Data} (Herzegh, 1988).  Those
calculations used by \suds\ which are not taken directly from this memorandum
are outlined below.

\section{Obtaining Dewpoint Temperature from Saturation Mixing 
	Ratio and Pressure}

The temperature at which a given constant mixing ratio line and isobar 
intersect is used in many places in \suds.  The formulas used to calculate
this intersection are:

\begin{eqnarray}
	T_d (^\circ{\rm K}) &=& \frac{T_3 A}
		{A - B + \sqrt{B^2 + 2A \ln(E_3/e_{sw})}} \label{eq-T}\\
	e_{sw} ({\rm mb}) &=& \frac{Pw_{sw}}{1000\epsilon + w_{sw}}\label{eq-e}
\end{eqnarray}

where $P$ is the pressure in mb, $w$ is the mixing ratio in g kg$^{-1}$, and
the other terms are constants, as follows:
\begin{eqnarray*}
	T_3 &=& 273.15 ^\circ{\rm K} \\
	E_3 &=& 6.1078 {\rm~mb} \\
	\epsilon &=& 0.622 \\
	A &=& 5.0065 \\
	B &=& 19.83923
\end{eqnarray*}

The derivation of Equation~\ref{eq-T} proceeds from Note~6 of Herzegh (1988),
using the formula given there and applying the fact that $T = T_d$ when 
$e = e_{sw}$:
\[
	e_{sw} = E_3 \exp\left[A \ln(T_3 / T_d)] 
		\exp[(A + B)(1 - T_3 / T_d)\right]
\]
Take the logarithm of both sides to obtain:
\[
	\ln e_{sw} = \ln E_3 + A \ln(T_3 / T_d) + (A + B)(1 - T_3 / T_d)
\]
Perform the substitution $x = T_3 / T_d$:
\[
	\ln e_{sw} = \ln E_3 + A \ln x + (A + B)(1 - x)
\]
Taylor series expand $\ln x~~(~= (x-1) - \frac{1}{2}(x-1)^2 + \cdots)$, and
substitute $-(x-1)$ for $(1-x)$:
\[
	\ln e_{sw} = \ln E_3 + A(x-1) - \frac{A}{2}(x-1)^2 + 
		{\rm smaller~terms} - (A+B)(x-1)
\]
Combining terms yields:
\[
	0 = \ln E_3 - \ln e_{sw} - B(x-1) - \frac{A}{2}(x-1)^2
\]
Perform the substitution $u = x - 1$ and simplify:
\[
	0 = -\frac{A}{2}u^2 - Bu + ln (E_3 / e_{sw})
\]
Solve for $u$, choosing the positive root:
\[
	u = \frac{-B + \sqrt{B^2 + 2A \ln (E_3 / e_{sw})}}{A}
\]
Solve for $x~~(~= 1 + u)$:
\[
	x = \frac{A - B + \sqrt{B^2 + 2A \ln (E_3 / e_{sw})}}{A}
\]
Finally, solve for $T_d~~(~= T_3 / x)$ to obtain Equation~\ref{eq-T}:
\[
	T_d = \frac{T_3 A}{A - B + \sqrt{B^2 + 2A \ln (E_3 / e_{sw})}}
\]

Equation~\ref{eq-e} is derived from Note~2 of Herzegh (1988), rearranging
the terms to express the saturation vapor pressure in terms of the saturation
mixing ratio:
\[
	w_{sw} = \epsilon\left[\frac{e_{sw}}{P-e_{sw}}\right]1000 
	~~~\Longrightarrow~~~
	e_{sw} = \frac{Pw_{sw}}{1000\epsilon + w_{sw}}
\]

\section {Saturation Vapor Pressure}
The equation for calculating saturation vapor pressure from Note~6 of Herzegh
(1988) is:
\[
	e_{sw} = E_3 \exp\left[A \ln(T_3/T) \right] 
		\exp\left[(A+B)(1-T_3/T)\right]
\]
Applying this equation involves the calculation of two exponentials and one
logarithm and these calculations are relatively compute-intensive.  Since the
saturation vapor pressure is calculated many times for many different points
by \suds, compute times become prohibitive if the exact equation is used every
time.  For this reason, and because temperature is the only free variable in 
the equation, calculation of saturation vapor pressure has been 
implemented as a look-up table with values stored for every 0.5$^\circ$K from
210.0$^\circ$K to 310.0$^\circ$K.  Simple linear interpolation is used between
the two temperatures nearest a given input point.  For input values which lie
outside the table, the exact formula is used.  The lookup table yields a
maximum relative error on the order of a few parts in 10$^4$.  For most end
calculations (e.g., temperature of the lifting condensation level), the 
errors introduced are on this same scale.  For calculations of energy, however,
the numeric integrations used compound the error, so relative errors may climb
to the order of 10$^{-2}$.

\section {Potential Temperature}

From Herzegh (1988), the equation for potential temperature is:
\[
	\theta = T \left[\frac{1000}{P}\right]^{(R_d/c_p)(1-0.00028w)}
\]
Application of this equation involves exponentiation; since \suds\ uses
this equation frequently, the compute time can be prohibitive if the exact 
form is used.  The equation has three free parameters, $T$, $P$, and $w$, so 
a simple look-up table cannot be used as was done for the saturation vapor 
pressure.  Instead, a relation was developed which is less compute-intensive
but is not an exact solution.  First, the mixing ratio term $(1-0.00028w)$ in
the exponent was dropped, due to its small effect (less than 1 part in $10^3$ 
in realistic situations).  Second, the substitution $u = 1000 / P$ was made, 
yielding:
\[
	\theta = Tu^{(R_d/c_p)}
\]
Next, the exponential in $u$ is expanded into a seven term Taylor series about
some value $u_0$:
\[
	\theta = T \sum_{i=0}^6 \left[\prod_{j=0}^{i-1}(R_d/c_p-j)\right]
		\frac{(u-u_0)^i}{i!} u_0^{R_d/c_p-i} + {\rm smaller~terms}
\]
Three values of $u_0$ were chosen: $1000 / 700$, $1000 / 350$, and 
$1000 / 175$, corresponding to pressures of 700~mb, 350~mb, and 175~mb, 
respectively.  The resulting formulas and the ranges over which they are 
applied are shown below:
\[
	\newcommand{\diff}{(u - u_0)}
	\theta~~(^\circ\rm K) = 
		\cases{	T\Big[1.10714599 + 0.22116497\diff - \cr
			~~~0.05531763\diff^2 + 0.02213146\diff^3 - \cr
			~~~0.01051376\diff^4 + 0.00546766\diff^5 - \cr
			~~~0.00300743\diff^6\Big] 
					& $u_0 = 1000 / 700$ \cr
					& if $P > 500$ mb; \cr
			\cr
			T\Big[1.34930719 + 0.13476972\diff - \cr
			~~~0.01685425\diff^2 + 0.00337152\diff^3 - \cr
			~~~0.00080084\diff^4 + 	0.00020824\diff^5 - 
					& \hskip 1.75 truein (A.3)
					  \hskip -2.0truein\cr
			~~~0.00005727\diff^6\Big]
					& $u_0 = 1000 / 350$ \cr
					& if $500{\rm~mb}\ge P > 250$ mb; \cr
			\cr
			T\Big[1.64443524 + 0.08212365\diff - \cr
			~~~0.00513518\diff^2 + 0.00051362\diff^3 - \cr
			~~~0.00006100\diff^4 + 0.00000793\diff^5 - \cr
			~~~0.00000109\diff^6\Big]
					& $u_0 = 1000 / 175$ \cr
					& if $250{\rm~mb}\ge P > 125$ mb; \cr
			\cr
			Tu^{R_d/c_p}	& otherwise.
			}
\]


\part{Field Names}
\label{app-fields}
\begin{center}
\begin{tabular}{|r|l|}
	\hline
	Name(s)				& Description	\\
	\hline \hline
	\tt alt, altitude		& Altitude MSL	\\
	\tt ascent, dz/dt, dz		& Balloon ascent rate \\
	\tt dp, dewpoint		& Dewpoint	\\
	\tt lat, latitude		& Latitude	\\
	\tt lon, longitude		& Longitude	\\
	\tt mflux\_uv			& Moisture flux w.r.t. u or v wind \\
	\tt mr				& Mixing ratio	\\
	\tt pres, press, pressure	& Pressure	\\
	\tt qpres			& Pressure quality	\\
	\tt qrh				& Relative humidity quality	\\
	\tt qtemp			& Temperature quality	\\
	\tt rh				& Relative humidity	\\
	\tt ri				& Richardson number \\
	\tt rtype			& NWS record type	\\
	\tt temp, temperature, tdry	& Temperature (dry bulb)\\
	\tt theta, pt			& Potential temperature	\\
	\tt theta\_e, ept		& Equivalent potential temperature \\
	\tt theta\_v, vpt		& Virtual potential temperature \\
	\tt time			& Time since sounding launch	\\
	\tt u\_wind			& U component of wind	\\
	\tt vt, t\_v			& virtual temperature	\\
	\tt v\_wind			& V component of wind	\\
	\tt wdir			& Wind direction	\\
	\tt wspd			& Wind speed	\\
	\tt u\_prime			& $\parallel$ wind component 
						({\tt xsect} only)	\\
	\tt v\_prime			& $\perp$ wind component 
						({\tt xsect} only)	\\
	\hline
\end{tabular}
\end{center}

\part{Color Names}
\label{color-names}
\begin{center}
    \scriptsize
    \baselineskip 12pt
    \vfill
    \begin{tabular}{c|c|c|c|c}
        snow			& dim gray                & LightSkyBlue                    & MediumSpringGreen       & DarkSalmon              \cr
        ghost white		& DimGray                 & steel blue                      & green yellow            & salmon                  \cr
        GhostWhite		& dim grey                & SteelBlue                       & GreenYellow             & light salmon            \cr
        white smoke		& DimGrey                 & light steel blue                & lime green              & LightSalmon             \cr
        WhiteSmoke		& slate gray              & LightSteelBlue                  & LimeGreen               & orange                  \cr
        gainsboro		& SlateGray               & light blue                      & yellow green            & dark orange             \cr
        floral white		& slate grey              & LightBlue                       & YellowGreen             & DarkOrange              \cr
        FloralWhite		& SlateGrey               & powder blue                     & forest green            & coral                   \cr
        old lace		& light slate gray        & PowderBlue                      & ForestGreen             & light coral             \cr
        OldLace			& LightSlateGray          & pale turquoise                  & olive drab              & LightCoral              \cr
        linen			& light slate grey        & PaleTurquoise                   & OliveDrab               & tomato                  \cr
        antique white		& LightSlateGrey          & dark turquoise                  & dark khaki              & orange red              \cr
        AntiqueWhite		& gray                    & DarkTurquoise                   & DarkKhaki               & OrangeRed               \cr
        papaya whip		& grey                    & medium turquoise                & khaki                   & red                     \cr
        PapayaWhip		& light grey              & MediumTurquoise                 & pale goldenrod          & hot pink                \cr
        blanched almond		& LightGrey               & turquoise                       & PaleGoldenrod           & HotPink                 \cr
        BlanchedAlmond		& light gray              & cyan                            & light goldenrod yellow  & deep pink               \cr
        bisque			& LightGray               & light cyan                      & LightGoldenrodYellow    & DeepPink                \cr
        peach puff		& midnight blue           & LightCyan                       & light yellow            & pink                    \cr
        PeachPuff		& MidnightBlue            & cadet blue                      & LightYellow             & light pink              \cr
        navajo white		& navy                    & CadetBlue                       & yellow                  & LightPink               \cr
        NavajoWhite		& navy blue               & medium aquamarine               & gold                    & pale violet red         \cr
        moccasin		& NavyBlue                & MediumAquamarine                & light goldenrod         & PaleVioletRed           \cr
        cornsilk		& cornflower blue         & aquamarine                      & LightGoldenrod          & maroon                  \cr
        ivory			& CornflowerBlue          & dark green                      & goldenrod               & medium violet red       \cr
        lemon chiffon		& dark slate blue         & DarkGreen                       & dark goldenrod          & MediumVioletRed         \cr
        LemonChiffon		& DarkSlateBlue           & dark olive green                & DarkGoldenrod           & violet red              \cr
        seashell		& slate blue              & DarkOliveGreen                  & rosy brown              & VioletRed               \cr
        honeydew		& SlateBlue               & dark sea green                  & RosyBrown               & magenta                 \cr
        mint cream		& medium slate blue       & DarkSeaGreen                    & indian red              & violet                  \cr
        MintCream		& MediumSlateBlue         & sea green                       & IndianRed               & plum                    \cr
        azure			& light slate blue        & SeaGreen                        & saddle brown            & orchid                  \cr
        alice blue		& LightSlateBlue          & medium sea green                & SaddleBrown             & medium orchid           \cr
        AliceBlue		& medium blue             & MediumSeaGreen                  & sienna                  & MediumOrchid            \cr
        lavender		& MediumBlue              & light sea green                 & peru                    & dark orchid             \cr
        lavender blush		& royal blue              & LightSeaGreen                   & burlywood               & DarkOrchid              \cr
        LavenderBlush		& RoyalBlue               & pale green                      & beige                   & dark violet             \cr
        misty rose		& blue                    & PaleGreen                       & wheat                   & DarkViolet              \cr
        MistyRose		& dodger blue             & spring green                    & sandy brown             & blue violet             \cr
        white			& DodgerBlue              & SpringGreen                     & SandyBrown              & BlueViolet              \cr
        black			& deep sky blue           & lawn green                      & tan                     & purple                  \cr
        dark slate gray		& DeepSkyBlue             & LawnGreen                       & chocolate               & medium purple           \cr
        DarkSlateGray		& sky blue                & green                           & firebrick               & MediumPurple            \cr
        dark slate grey		& SkyBlue                 & chartreuse                      & brown                   & thistle                 \cr
        DarkSlateGrey		& light sky blue          & medium spring green             & dark salmon             & snow1                   \cr
    \end{tabular}
    \eject
    \begin{tabular}{c|c|c|c|c}
        snow2                   & DodgerBlue3             & SeaGreen4                       & burlywood1              & pink2                   \cr
        snow3                   & DodgerBlue4             & PaleGreen1                      & burlywood2              & pink3                   \cr
        snow4                   & SteelBlue1              & PaleGreen2                      & burlywood3              & pink4                   \cr
        seashell1               & SteelBlue2              & PaleGreen3                      & burlywood4              & LightPink1              \cr
        seashell2               & SteelBlue3              & PaleGreen4                      & wheat1                  & LightPink2              \cr
        seashell3               & SteelBlue4              & SpringGreen1                    & wheat2                  & LightPink3              \cr
        seashell4               & DeepSkyBlue1            & SpringGreen2                    & wheat3                  & LightPink4              \cr
        AntiqueWhite1           & DeepSkyBlue2            & SpringGreen3                    & wheat4                  & PaleVioletRed1          \cr
        AntiqueWhite2           & DeepSkyBlue3            & SpringGreen4                    & tan1                    & PaleVioletRed2          \cr
        AntiqueWhite3           & DeepSkyBlue4            & green1                          & tan2                    & PaleVioletRed3          \cr
        AntiqueWhite4           & SkyBlue1                & green2                          & tan3                    & PaleVioletRed4          \cr
        bisque1                 & SkyBlue2                & green3                          & tan4                    & maroon1                 \cr
        bisque2                 & SkyBlue3                & green4                          & chocolate1              & maroon2                 \cr
        bisque3                 & SkyBlue4                & chartreuse1                     & chocolate2              & maroon3                 \cr
        bisque4                 & LightSkyBlue1           & chartreuse2                     & chocolate3              & maroon4                 \cr
        PeachPuff1              & LightSkyBlue2           & chartreuse3                     & chocolate4              & VioletRed1              \cr
        PeachPuff2              & LightSkyBlue3           & chartreuse4                     & firebrick1              & VioletRed2              \cr
        PeachPuff3              & LightSkyBlue4           & OliveDrab1                      & firebrick2              & VioletRed3              \cr
        PeachPuff4              & SlateGray1              & OliveDrab2                      & firebrick3              & VioletRed4              \cr
        NavajoWhite1            & SlateGray2              & OliveDrab3                      & firebrick4              & magenta1                \cr
        NavajoWhite2            & SlateGray3              & OliveDrab4                      & brown1                  & magenta2                \cr
        NavajoWhite3            & SlateGray4              & DarkOliveGreen1                 & brown2                  & magenta3                \cr
        NavajoWhite4            & LightSteelBlue1         & DarkOliveGreen2                 & brown3                  & magenta4                \cr
        LemonChiffon1           & LightSteelBlue2         & DarkOliveGreen3                 & brown4                  & orchid1                 \cr
        LemonChiffon2           & LightSteelBlue3         & DarkOliveGreen4                 & salmon1                 & orchid2                 \cr
        LemonChiffon3           & LightSteelBlue4         & khaki1                          & salmon2                 & orchid3                 \cr
        LemonChiffon4           & LightBlue1              & khaki2                          & salmon3                 & orchid4                 \cr
        cornsilk1               & LightBlue2              & khaki3                          & salmon4                 & plum1                   \cr
        cornsilk2               & LightBlue3              & khaki4                          & LightSalmon1            & plum2                   \cr
        cornsilk3               & LightBlue4              & LightGoldenrod1                 & LightSalmon2            & plum3                   \cr
        cornsilk4               & LightCyan1              & LightGoldenrod2                 & LightSalmon3            & plum4                   \cr
        ivory1                  & LightCyan2              & LightGoldenrod3                 & LightSalmon4            & MediumOrchid1           \cr
        ivory2                  & LightCyan3              & LightGoldenrod4                 & orange1                 & MediumOrchid2           \cr
        ivory3                  & LightCyan4              & LightYellow1                    & orange2                 & MediumOrchid3           \cr
        ivory4                  & PaleTurquoise1          & LightYellow2                    & orange3                 & MediumOrchid4           \cr
        honeydew1               & PaleTurquoise2          & LightYellow3                    & orange4                 & DarkOrchid1             \cr
        honeydew2               & PaleTurquoise3          & LightYellow4                    & DarkOrange1             & DarkOrchid2             \cr
        honeydew3               & PaleTurquoise4          & yellow1                         & DarkOrange2             & DarkOrchid3             \cr
        honeydew4               & CadetBlue1              & yellow2                         & DarkOrange3             & DarkOrchid4             \cr
        LavenderBlush1          & CadetBlue2              & yellow3                         & DarkOrange4             & purple1                 \cr
        LavenderBlush2          & CadetBlue3              & yellow4                         & coral1                  & purple2                 \cr
        LavenderBlush3          & CadetBlue4              & gold1                           & coral2                  & purple3                 \cr
        LavenderBlush4          & turquoise1              & gold2                           & coral3                  & purple4                 \cr
        MistyRose1              & turquoise2              & gold3                           & coral4                  & MediumPurple1           \cr
        MistyRose2              & turquoise3              & gold4                           & tomato1                 & MediumPurple2           \cr
        MistyRose3              & turquoise4              & goldenrod1                      & tomato2                 & MediumPurple3           \cr
        MistyRose4              & cyan1                   & goldenrod2                      & tomato3                 & MediumPurple4           \cr
        azure1                  & cyan2                   & goldenrod3                      & tomato4                 & thistle1                \cr
        azure2                  & cyan3                   & goldenrod4                      & OrangeRed1              & thistle2                \cr
        azure3                  & cyan4                   & DarkGoldenrod1                  & OrangeRed2              & thistle3                \cr
        azure4                  & DarkSlateGray1          & DarkGoldenrod2                  & OrangeRed3              & thistle4                \cr
        SlateBlue1              & DarkSlateGray2          & DarkGoldenrod3                  & OrangeRed4              & gray0                   \cr
        SlateBlue2              & DarkSlateGray3          & DarkGoldenrod4                  & red1                    & gray1                   \cr
        SlateBlue3              & DarkSlateGray4          & RosyBrown1                      & red2                    & $\cdot$                 \cr
        SlateBlue4              & aquamarine1             & RosyBrown2                      & red3                    & $\cdot$                 \cr
        RoyalBlue1              & aquamarine2             & RosyBrown3                      & red4                    & $\cdot$                 \cr
        RoyalBlue2              & aquamarine3             & RosyBrown4                      & DeepPink1               & gray99                  \cr
        RoyalBlue3              & aquamarine4             & IndianRed1                      & DeepPink2               & gray100                 \cr
        RoyalBlue4              & DarkSeaGreen1           & IndianRed2                      & DeepPink3               & grey0                   \cr
        blue1                   & DarkSeaGreen2           & IndianRed3                      & DeepPink4               & grey1                   \cr
        blue2                   & DarkSeaGreen3           & IndianRed4                      & HotPink1                & $\cdot$                 \cr
        blue3                   & DarkSeaGreen4           & sienna1                         & HotPink2                & $\cdot$                 \cr
        blue4                   & SeaGreen1               & sienna2                         & HotPink3                & $\cdot$                 \cr
        DodgerBlue1             & SeaGreen2               & sienna3                         & HotPink4                & grey99                  \cr
        DodgerBlue2             & SeaGreen3               & sienna4                         & pink1                   & grey100                 \cr
    \end{tabular}
\end{center}


\part{References}

\newcommand{\reference}{\penalty-1000\hangindent = 2em\hangafter = 1}

\reference
Foote, G. B., 1984: Influences of gust fronts on the propagation of storms. 
{\sl Proc. 9th International Cloud Physics Conf.}, Tallinn Estonia 
International Association of Meteorology and Atmospheric Physics.  419--421.

\reference
Herzegh, P. H., 1988: Formulation of output parameters for PAM II CMF data.
NCAR/FOF informal publication.

\reference
Wilson, J. W., and C. K. Mueller, 1987: Thunderstorm nowcasting using Doppler
radar and soundings.  {\sl Proc. Symp. Mesoscale Analysis and Forecasting},
Vancouver, 55--60.

\part{Index}
{ % begin special index stuff
\makeatletter
\gdef\thepage{Index--\the\c@page}

\catcode`\@=\other

\catcode`\% = \other	\catcode`\& = \other	\catcode`\# = \other
\catcode`\$ = \other	\catcode`\_ = \other	\catcode`\~ = \other

\input{sudsndx.dats}
} % end special index stuff

\end{document}
