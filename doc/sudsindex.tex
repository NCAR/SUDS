% Index generation facilities
%
% An index file <TeXfile>ndx.dat is created which contains entries 
% of the form:
%	\entry {sortstring}{page}{topic}
%    or
%	\entry {sortstring}{page}{topic}{subtopic}
%
% The texindex program is run on this file and creates a file <TeXfile>ndx.dats
% (the sorted index file) which has lines like:
%	\initial {c}			before the topics whose initial is c
%
%	\entry {topic}{pagelist}	for a topic that is used without 
%					subtopics
%
%	\primary {topic}		for the beginning of a topic that is 
%					used with subtopics
%
%	\secondary {subtopic}{pagelist}	for each subtopic
%

%
% Open file <current_file>.ndx where <current_file> is the name of
% the file we are now TeXing.
%
\newif\ifindexopen
\indexopenfalse

\newwrite \ndxfileout

%
% The "user-level" commands
%
% \doindex		- Start the index
% \index{something}	- Create an index entry for "something"
% \mainindex{something}	- Make the main index entry for "something", i.e.
%			  the page number will be in italics
% \subindex{main}{sub}	- Create a subentry for "sub" under the main
%			  entry "main"
% \mainsubindex{main}{sub}
%			- Main entry for a subindexed item
%

\def\doindex{\openout \ndxfileout = \jobname ndx.dat
	\indexopentrue}

\def\index#1{\ndxentry{#1}{#1}\ignorespaces}
\def\mainindex#1{\ndxmainentry{#1}{#1}\ignorespaces}

\def\subindex#1#2{\subndxentry{#1}{#2}{#2}\ignorespaces}
\def\mainsubindex#1#2{\subndxmainentry{#1}{#2}{#2}\ignorespaces}


%
% Define \realbackslash to give us a printable backslash
%
{\catcode`\@=0 \catcode`\\=\other
	@gdef@realbackslash{\}}

%
% \ndxentry creates a new index entry 
%	argument 1:	the sort string (e.g. "routine ()")
%	argument 2:	the output string (e.g. "\tt routine ()")
%
\def\ndxentry#1#2{\ifindexopen%
\write \ndxfileout {\realbackslash entry {#1}{\thepage}{#2}}%
\if@nobreak\ifvmode\nobreak\fi\fi\fi}
\def\ndxmainentry#1#2{\ifindexopen%
\write \ndxfileout {\realbackslash entry {#1}{{\it\thepage}}{#2}}%
\if@nobreak\ifvmode\nobreak\fi\fi\fi}

%
% \subndxentry works like \ndxentry except it creates a subentry for a main
% index entry
%	argument 1:	primary entry (e.g. "Graphics")
%	argument 2:	subentry sort string (e.g. "G_redraw ()")
%	argument 3:	subentry output string (e.g. "\tt G_redraw ()")
%
% NOTE: it is important to leave " !" between #1 and #2 in the
%	definitions below for the secondary entries to be sorted properly
%
\def\subndxentry#1#2#3{\ifindexopen%
  \write \ndxfileout {\realbackslash entry {#1 !#2}{\thepage}{#1}{#3}}%
\if@nobreak\ifvmode\nobreak\fi\fi\fi}
\def\subndxmainentry#1#2#3{\ifindexopen%
  \write \ndxfileout {\realbackslash entry {#1 !#2}{{\it\thepage}}{#1}{#3}}%
\if@nobreak\ifvmode\nobreak\fi\fi\fi}

%
% Definition for the index line in the TOC
%
\def\tocindexline{\ifindexopen\contentsline{chapter}{Index}{}\fi}

\def\l@index#1#2{\pagebreak[3]
 \vskip 1.0em plus 1pt \@tempdima 1.5em \begingroup
 \parindent \z@ \rightskip \@pnumwidth
 \parfillskip -\@pnumwidth
 \bf \leavevmode #1\hfil 
 \hbox to\@pnumwidth{\hfil}
 \par
 \endgroup}


%
% Define the macros used in formatting output of the sorted index material.
%
% These macros are used by the sorted index file itself.
% Change them to control the appearance of the index.
%
\font\initialfont = cmssbx10 scaled \magstep2

\def\ptexline{\hbox to \hsize}

\def\Dotsbox{\hbox to 1em{\hss.\hss}}

\outer\def\initial #1{
	\bigbreak
	\ptexline{\initialfont #1 \hfill}
	\kern 2pt
	\penalty3000}

\outer\def\entry #1#2{{
	\parfillskip = 0in plus 1 fil
	\parskip = 0in
	\parindent = 0in
	\hangindent = 1cm
	\hangafter = 1
	\noindent{#1}, #2\par}}

\def\primary #1{\ptexline{#1\hfil}\nobreak}

\newskip\secondaryindent \secondaryindent=0.5cm

\def\secondary #1#2{{
	\parfillskip = 0in plus 1 fil
	\parskip = 0in
	\hangindent = 1.5cm
	\hangafter = 1
	\noindent
	\hskip\secondaryindent{#1}, #2\par}}

%
% Hacked up "onepageout" macro.
%

\def\xonepageout#1{
	\setbox\@outputbox = \xpagebody{#1}
	\@outputpage}
%\def\xonepageout#1{
%	\shipout\vbox{\xpagebody{#1}}
%	\addtocounter{page}{1}
%
%	\ifnum \outputpenalty > -20000 
%		% do nothing
%	\else
%		\dosupereject
%	\fi}

\def\xpagebody#1{\vbox to \textheight{\boxmaxdepth = \maxdepth #1}}

%
% Define two-column mode, which is used in indexes.
% Adapted from the TeXBook, page 416
%

\newbox\partialpage

\newdimen\doublecolumnhsize
\doublecolumnhsize = 0.5\textwidth
\advance\doublecolumnhsize by -0.1 true in

\newdimen\doublecolumnvsize
\doublecolumnvsize = 2\textheight
\advance\doublecolumnvsize by 0.5 true in

\def\doublecolumns{
	\output = {\global\setbox\partialpage = 
		\vbox{\unvbox255\kern -\topskip \kern \baselineskip}}
	\eject
	\output = {\doublecolumnout}
	\hsize = \doublecolumnhsize
	\vsize = \doublecolumnvsize}

\def\enddoublecolumns{\output = {\balancecolumns}
	\eject 
	\global\pagegoal = \vsize}

\def\doublecolumnout{
	\splittopskip = \topskip 
	\splitmaxdepth = \maxdepth
	\dimen@ = \textheight
	\advance \dimen@ by -\ht \partialpage
	\setbox0 = \vsplit255 to \dimen@
	\setbox2 = \vsplit255 to \dimen@
	\xonepageout \pagesofar 
	\unvbox255 
	\penalty \outputpenalty}
  
\def\pagesofar{\unvbox\partialpage
	\wd0 = \hsize
	\wd2 = \hsize
	\hbox to \textwidth {\box0\hfil\box2}}

\def\balancecolumns{
	\setbox0 = \vbox{\unvbox255}
	\dimen@ = \ht0
	\advance \dimen@ by \topskip
	\advance \dimen@ by -\baselineskip
	\divide \dimen@ by 2
	\splittopskip = \topskip
	{\vbadness = 10000
	\loop 
		\global\setbox3 = \copy0
		\global\setbox1 = \vsplit3 to \dimen@
	\ifdim \ht3 > \dimen@ \global\advance \dimen@ by 1pt \repeat}
	\setbox0 = \vbox to \dimen@ {\unvbox1}
	\setbox2 = \vbox to \dimen@ {\unvbox3}
	\pagesofar}
